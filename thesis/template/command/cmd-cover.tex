%
% This file is part of the project of
% National Cheng Kung University (NCKU) Thesis/Dissertation Template in LaTex.
% This project is hold at
%     <https://github.com/wengan-li/ncku-thesis-template-latex>
% by Wen-Gan Li.
%
% This project is distributed in the hope of usefuling to someone,
% you can redistribute it and/or modify it under the terms of the
% Attribution-NonCommercial-ShareAlike 4.0 International.
%
% You should have received a copy of the
% Attribution-NonCommercial-ShareAlike 4.0 International
% along with this project.
% If not, see <http://creativecommons.org/licenses/by-nc-sa/4.0/legalcode.txt>.
%
% Please feel free to fork it, modify it, and try it.
% Have fun !!!
%

% Some helper function use in cover

% ----------------------------------------------------------------------------
\newcommand{\StartCover}
{
  %
  \singlespacing%
  %
  \StartNewPage
  %
  % 設定使用 無頁碼
  \thispagestyle{empty}
  %
  \EnableCoverPageStyle
  %
  % Aligned to the center of the page
  \begin{center}
} % End of \newcommand{}

\newcommand{\EndCover}
{
  % End of alignment
  \end{center}
  \DisableCoverPageStyle
  \EndOfPage
  \UseDefaultLineStretch
} % End of \newcommand{}
% ----------------------------------------------------------------------------

% --- University name 學校名字 ---
\newcommand\VarUniversityChiName{國立成功大學}           % Default
\newcommand\VarUniversityEngName{National Cheng Kung University} % Default

\newcommand{\SetUniversityChiName}[1]{%
  \renewcommand{\VarUniversityChiName}{#1}%
} % End of \newcommand{}

\newcommand{\SetUniversityEngName}[1]{%
  \renewcommand{\VarUniversityEngName}{#1}%
} % End of \newcommand{}

\newcommand{\SetUniversityName}[2]
{%
  \SetUniversityChiName{#1}%
  \SetUniversityEngName{#2}%
} % End of \newcommand{}

\newcommand{\GetUniversityChiName}{\VarUniversityChiName}
\newcommand{\GetUniversityEngName}{\VarUniversityEngName}
% ----------------------------------------------------------------------------

% --- Chinese / English title 中英文論文題目 ---
\newcommand{\VarThesisChiName}{Chinese Title Here} % Default
\newcommand{\VarThesisEngName}{English Title Here} % Default
\newcommand{\SetChiTitle}[1]{\renewcommand{\VarThesisChiName}{#1}}
\newcommand{\SetEngTitle}[1]{\renewcommand{\VarThesisEngName}{#1}}
\newcommand{\SetTitle}[2]
{
  \SetChiTitle{#1}
  \SetEngTitle{#2}
} % End of \newcommand{}

\newcommand{\GetChiTitle}{\VarThesisChiName}
\newcommand{\GetEngTitle}{\VarThesisEngName}

% ----------------------------------------------------------------------------

% --- User's name 使用者名字 ---
\newcommand{\VarMyChiName}{你的名字}     % Default
\newcommand{\VarMyEngName}{Your name}   % Default
\newcommand{\SetMyChiName}[1]{\renewcommand{\VarMyChiName}{#1}}
\newcommand{\SetMyEngName}[1]{\renewcommand{\VarMyEngName}{#1}}
\newcommand{\SetMyName}[2]
{
  \SetMyChiName{#1}
  \SetMyEngName{#2}
} % End of \newcommand{}

\newcommand{\GetAuthorChiName}{\VarMyChiName}
\newcommand{\GetAuthorEngName}{\VarMyEngName}

% ----------------------------------------------------------------------------

% --- Degree name 學位 ---
% thesis 是指論文的通稱
% dissertation 指的是博士的論文

% 碩士論文  Master's thesis
% 博士論文  Doctoral dissertation

\newcommand{\ValueDegreeMaster}{0}
\newcommand{\ValueDegreePhd}{1}
\newcommand{\FlagDegreeType}{\ValueDegreePhd} % Default
\newcommand{\GetFlagDegreeType}{\FlagDegreeType}
\newcommand{\SetFlagDegreeType}[1]{\renewcommand{\FlagDegreeType}{#1}}

\newcommand{\VarDegreeChiName}{碩士/博士} % Default
\newcommand{\VarDegreeEngName}{Master / Doctor} % Default
\newcommand{\degreeThesisEname}{Master's Thesis / Doctoral Dissertation} % Default

\newcommand{\GetChiDegree}{\VarDegreeChiName}
\newcommand{\GetEngDegree}{\VarDegreeEngName}
\newcommand{\GetEngDegreeThesis}{\degreeThesisEname}
\newcommand{\SetChiDegree}[1]{\renewcommand{\VarDegreeChiName}{#1}}
\newcommand{\SetEngDegree}[1]{\renewcommand{\VarDegreeEngName}{#1}}
\newcommand{\SetEngDegreeThesis}[1]{\renewcommand{\degreeThesisEname}{#1}}

\newcommand{\PhdDegree}
{
  \SetFlagDegreeType{\ValueDegreePhd}
  \SetChiDegree{博士}
  \SetEngDegree{Doctor}
  \SetEngDegreeThesis{Doctoral Dissertation}
} % End of \newcommand{}

\newcommand{\MasterDegree}
{
  \SetFlagDegreeType{\ValueDegreeMaster}
  \SetChiDegree{碩士}
  \SetEngDegree{Master}
  \SetEngDegreeThesis{Master's Thesis}
} % End of \newcommand{}

% ----------------------------------------------------------------------------

% --- Date 日期 ---

% \CoverDateNumInChi: 日期使用中文數字,
% 而不是阿拉伯數字,
% 故使用'\CoverDateNumInChi'可以顯示
% '第一章' 而不是 '中華民國 103 年 12 月 31 日'.
% \CoverDateNumInChi必須配合\DisplayCoverInChi來使用, 否則會無效.
%\CoverDateNumInChi

% --- 論文的日期 ---
\newcommand{\ThesisYear}{2014}  % Default
\newcommand{\ThesisMonth}{1}    % Default

\newcommand{\SetThesisDate}[2]{\SetThesisDate{#1}{#2}} % For backporting
\newcommand{\SetCoverDate}[2]
{
  \SetThesisTaiwanYear{#1}
  \renewcommand{\ThesisYear}{#1}
  \renewcommand{\ThesisMonth}{#2}
} % End of \newcommand{}

\newcommand{\GetThesisYear}{\ThesisYear}
\newcommand{\GetThesisYearInTaiwanYear}{\ThesisTaiwanYearResult}
\newcommand{\GetThesisMonth}{\ThesisMonth}
\newcommand{\GetThesisMonthNumInChi}{\zhnumber{\ThesisMonth}}
\newcommand{\GetThesisMonthInEng}{\GetMonthInEng{\ThesisMonth}}

% ---  口試的日期 ---
\newcommand{\OralChiYear}{101}      % Default
\newcommand{\OralChiMonth}{1}       % Default
\newcommand{\OralChiDay}{1}         % Default
\newcommand{\OralEngYear}{2014}     % Default
\newcommand{\OralEngMonth}{January} % Default
\newcommand{\OralEngDay}{1}         % Default

\newcommand{\GetOralChiYear}{\OralChiYear}
\newcommand{\GetOralYearInTaiwanYear}
{\SetThesisTaiwanYear{\OralEngYear}\ThesisTaiwanYearResult}
\newcommand{\GetOralYearInTaiwanYearNumInChi}
{\SetThesisTaiwanYear{\OralEngYear}\zhdigits{\ThesisTaiwanYearResult}}
\newcommand{\GetOralChiMonth}{\OralChiMonth}
\newcommand{\GetOralChiDay}{\OralChiDay}
\newcommand{\GetOralEngYear}{\OralEngYear}
\newcommand{\GetOralEngMonth}{\OralEngMonth}
\newcommand{\GetOralEngDay}{\OralEngDay}
\newcommand{\GetOralEngDayNumInChi}{\zhnumber{\OralEngDay}}

\newcommand{\SetOralChiDate}[3]
{
  \SetOralTaiwanYear{#1}
  \renewcommand{\OralChiYear}{\OralTaiwanYearResult}
  \renewcommand{\OralChiMonth}{#2}
  \renewcommand{\OralChiDay}{#3}
} % End of \newcommand{}

\newcommand{\SetOralEngDate}[3]
{
  \renewcommand{\OralEngYear}{#1}
  \renewcommand{\OralEngMonth}{\GetMonthInEng{#2}}
  \renewcommand{\OralEngDay}{#3}
} % End of \newcommand{}

\newcommand{\SetOralDate}[3]
{
  \SetOralChiDate{#1}{#2}{#3}
  \SetOralEngDate{#1}{#2}{#3}
} % End of \newcommand{}

% ----------------------------------------------------------------------------

% --- 學院 College, 系所 Department and Institute ---

% --------------------------- College ---------------------------
\newcommand{\VarCollegeChiName}{學院 C}
\newcommand{\VarCollegeEngName}{College of C}
\newcommand{\SetCollChiName}[1]{\renewcommand{\VarCollegeChiName}{#1}}
\newcommand{\SetCollEngName}[1]{\renewcommand{\VarCollegeEngName}{#1}}
\newcommand{\SetCollName}[2]
{
  \SetCollChiName{#1}
  \SetCollEngName{#2}
} % End of \newcommand{}

\newcommand{\GetCollChiName}{\VarCollegeChiName}
\newcommand{\GetCollEngName}{\VarCollegeEngName}

% --------------------------- Department ---------------------------
\newcommand{\VarDepartmentChiName}{A 系 / 所}
%\newcommand{\VarDepartmentEngName}{DeptA} % Short form of department
\newcommand{\VarDepartmentEngFullName}{Department / Insitute A} % Full name of department
\newcommand{\SetDeptChiName}[1]{\renewcommand{\VarDepartmentChiName}{#1}}
%\newcommand{\SetDeptEngShortName}[1]{\renewcommand{\VarDepartmentEngName}{#1}}
\newcommand{\SetDeptEngFullName}[1]{\renewcommand{\VarDepartmentEngFullName}{#1}}
\newcommand{\SetDeptName}[3]
{
  \SetDeptChiName{#1}
%  \SetDeptEngShortName{#2}
  \SetDeptEngFullName{#3}
} % End of \newcommand{}

\newcommand{\GetDeptChiName}{\VarDepartmentChiName}
\newcommand{\GetDeptEngName}{\VarDepartmentEngFullName}

% ----------------------------------------------------------------------------

% --- 指導老師 Advisor(s) ---
% 在封面上預算了最多3位的空間
% 中文名字固定以 博士 結尾
% 英文名字固定以 Dr. 開頭

\newcommand{\VarAdvisorChiNameA}{X}
\newcommand{\VarAdvisorEngNameA}{X}
\newcommand{\VarAdvisorChiNameB}{}
\newcommand{\VarAdvisorEngNameB}{}
\newcommand{\VarAdvisorChiNameC}{}
\newcommand{\VarAdvisorEngNameC}{}

\newcommand{\GetAdvisorChiNameA}{\VarAdvisorChiNameA}
\newcommand{\GetAdvisorEngNameA}{\VarAdvisorEngNameA}
\newcommand{\GetAdvisorChiNameB}{\VarAdvisorChiNameB}
\newcommand{\GetAdvisorEngNameB}{\VarAdvisorEngNameB}
\newcommand{\GetAdvisorChiNameC}{\VarAdvisorChiNameC}
\newcommand{\GetAdvisorEngNameC}{\VarAdvisorEngNameC}

\newcommand{\SetAdvisorChiNameA}[1]{\renewcommand{\VarAdvisorChiNameA}{#1}}
\newcommand{\SetAdvisorEngNameA}[1]{\renewcommand{\VarAdvisorEngNameA}{#1}}
\newcommand{\SetAdvisorChiNameB}[1]{\renewcommand{\VarAdvisorChiNameB}{#1}}
\newcommand{\SetAdvisorEngNameB}[1]{\renewcommand{\VarAdvisorEngNameB}{#1}}
\newcommand{\SetAdvisorChiNameC}[1]{\renewcommand{\VarAdvisorChiNameC}{#1}}
\newcommand{\SetAdvisorEngNameC}[1]{\renewcommand{\VarAdvisorEngNameC}{#1}}

\newcommand{\SetAdvisorNameA}[2]
{
  \SetAdvisorChiNameA{#1}
  \SetAdvisorEngNameA{#2}
} % End of \newcommand{}

\newcommand{\SetAdvisorNameB}[2]
{
  \SetAdvisorChiNameB{#1}
  \SetAdvisorEngNameB{#2}
} % End of \newcommand{}

\newcommand{\SetAdvisorNameC}[2]
{
  \SetAdvisorChiNameC{#1}
  \SetAdvisorEngNameC{#2}
} % End of \newcommand{}

% ----------------------------------------------------------------------------

% Use to create cover
\newcommand{\CreateCover}%
{
  \begin{document}
  % ------------------------------------------------
% This file is root file to build cover
% and shouldn't need to do any modify by design.
% 這檔案為論文封面的主檔案, 在設計上是不需要做任何的修改.
% ------------------------------------------------

% 基本設定 Basic configuration
%
% This file is part of the project of
% National Cheng Kung University (NCKU) Thesis/Dissertation Template in LaTex.
% This project is hold at
%     <https://github.com/wengan-li/ncku-thesis-template-latex>
% by Wen-Gan Li.
%
% This project is distributed in the hope of usefuling to someone,
% you can redistribute it and/or modify it under the terms of the
% Attribution-NonCommercial-ShareAlike 4.0 International.
%
% You should have received a copy of the
% Attribution-NonCommercial-ShareAlike 4.0 International
% along with this project.
% If not, see <http://creativecommons.org/licenses/by-nc-sa/4.0/legalcode.txt>.
%
% Please feel free to fork it, modify it, and try it.
% Have fun !!!
%

% ------------------------------------------------

\documentclass[12pt, a4paper, onecolumn, english]{report}

% ------------------------------------------------

% XeLaTex檢查點, 以要求必須使用XeLaTex來處理模版
\usepackage{ifxetex}
\ifxetex\else\errmessage{模版: 請使用XeLaTex來產生論文.}\stop\fi

% ------------------------------------------------

% 引用字體的基本設定
%
% No longer need \usepackage[T1]{fontenc} and
% \usepackage[utf8]{inputenc} when using XeLaTeX and LuaLaTeX as the engine.
%

% 引用fontspec以提供控制英文字型
\usepackage{fontspec}
\defaultfontfeatures{Ligatures=TeX} % To support LaTeX quoting style

% 引用xeCJK以提供控制中文字型
\usepackage{xeCJK}

% ------------------------------------------------

% 引用需要的LaTex packages

% Some base packages
\usepackage{geometry}
\usepackage{fp}
\usepackage{ifthen}
\usepackage{pgfkeys}
\usepackage{xparse}
\usepackage{amsmath}
\usepackage[framemethod=tikz]{mdframed}
\usepackage{url}
\usepackage{color}
\usepackage{etoolbox}

% For floats
% flafter package will make sure that the floats are
% not placed before their definition
\usepackage{flafter}

% For list
% Ref: <http://ftp.yzu.edu.tw/CTAN/macros/latex/contrib/enumitem/enumitem.pdf>
%\usepackage{enumitem}
%\setlist{noitemsep, nosep}

% For paragraphs
\usepackage{parskip}

% For line spacing
% Ref: <https://en.wikibooks.org/wiki/LaTeX/Paragraph_Formatting>
\usepackage{setspace}

% For PDF
\usepackage{hyperref}
\usepackage{pdfpages}

% For figure
\usepackage{graphicx}
\usepackage{caption}
\usepackage{subcaption}

% For table
\usepackage{array}
\usepackage{multirow}
\usepackage{booktabs}
\usepackage{diagbox}

% For comment
\usepackage{comment}

% For 目錄
\usepackage[tocgraduated]{tocstyle}
\usetocstyle{standard}
%\setcounter{tocdepth}{4} % 目錄會顯示subsubsection

% For chinese number in title
% http://ftp.yzu.edu.tw/CTAN/macros/latex/contrib/zhnumber/zhnumber.pdf
\usepackage{zhnumber}

% For pseudocode
\usepackage{algorithm}
\usepackage[noend]{algpseudocode}
\algnewcommand\algorithmicswitch{\textbf{switch}}
\algnewcommand\algorithmiccase{\textbf{case}}
\algnewcommand\algorithmicdefault{\textbf{default}}
\algnewcommand\algorithmicbreak{\textbf{break}}
\algdef{SE}[SWITCH]{Switch}{EndSwitch}[1]{\algorithmicswitch\ #1\ }{\algorithmicend\ \algorithmicswitch}%
\algdef{SE}[CASE]{Case}{EndCase}[1]{\algorithmiccase\ #1:}{\algorithmicend\ \algorithmiccase}%
\algdef{SE}[CASE]{Default}{EndDefault}[0]{\algorithmicdefault:}{\algorithmicend\ \algorithmiccase}%
\algtext*{EndSwitch}%
\algtext*{EndCase}%
\algtext*{EndDefault}%
\def\Break{\algorithmicbreak}

% For theorem
\usepackage{amsthm}
\usepackage{amssymb}
%\usepackage{chngcntr}
%\usepackage{./template/libs/apptools}

% ------------------------------------------------

% 有關學校對論文要求的設定

% --------------------------

% 一些用來設定function和variable的command
%
% This file is part of the project of
% National Cheng Kung University (NCKU) Thesis/Dissertation Template in LaTex.
% This project is hold at
%     <https://github.com/wengan-li/ncku-thesis-template-latex>
% by Wen-Gan Li.
%
% This project is distributed in the hope of usefuling to someone,
% you can redistribute it and/or modify it under the terms of the
% Attribution-NonCommercial-ShareAlike 4.0 International.
%
% You should have received a copy of the
% Attribution-NonCommercial-ShareAlike 4.0 International
% along with this project.
% If not, see <http://creativecommons.org/licenses/by-nc-sa/4.0/legalcode.txt>.
%
% Please feel free to fork it, modify it, and try it.
% Have fun !!!
%

% ----------------------------------------------------------------------------
% 一些用來設定function和variable的command
% Some function and variable that let user use and configure
%
% 此處只是一些預設值和function
% 修改內容是在'conf/conf'
% ----------------------------------------------------------------------------

% Static variable and some provided API
\input{./template/command/cmd-common}
\input{./template/command/cmd-figure}
\input{./template/command/cmd-figures}
\input{./template/command/cmd-table}
\input{./template/command/cmd-oral}
\input{./template/command/cmd-equation}
\input{./template/command/cmd-ref}
\input{./template/command/cmd-keyword}
\input{./template/command/cmd-pdf}
\input{./template/command/cmd-font}
\input{./template/command/cmd-numbering}
\input{./template/command/cmd-list}
\input{./template/command/cmd-spacing}

% Helper function for different page or chapter
\input{./template/command/cmd-thesis}
\input{./template/command/cmd-page}
\input{./template/command/cmd-cover}
\input{./template/command/cmd-chapter}
\input{./template/command/cmd-abstract}
\input{./template/command/cmd-abstract-extended}
\input{./template/command/cmd-acknowledgments}
\input{./template/command/cmd-appendix}
\input{./template/command/cmd-bibliography}
\input{./template/command/cmd-index}
\input{./template/command/cmd-nomenclature}
\input{./template/command/cmd-theorem}

% Some function that use for school
\input{./template/command/cmd-college}
\input{./template/command/cmd-department}
\input{./template/command/cmd-watermark}

% ----------------------------------------------------------------------------


% --------------------------

% 學校排版 Arrangement style
%
% This file is part of the project of
% National Cheng Kung University (NCKU) Thesis/Dissertation Template in LaTex.
% This project is hold at
%     <https://github.com/wengan-li/ncku-thesis-template-latex>
% by Wen-Gan Li.
%
% This project is distributed in the hope of usefuling to someone,
% you can redistribute it and/or modify it under the terms of the
% Attribution-NonCommercial-ShareAlike 4.0 International.
%
% You should have received a copy of the
% Attribution-NonCommercial-ShareAlike 4.0 International
% along with this project.
% If not, see <http://creativecommons.org/licenses/by-nc-sa/4.0/legalcode.txt>.
%
% Please feel free to fork it, modify it, and try it.
% Have fun !!!
%

% ------------------------------------------------

% 學校排版 Arrangement style

% 國立成功大學排版設定
\input{./template/style/ncku/ncku}

% ----------------------

% 自定的排版設定
%   請參考'template/style/ncku'的檔案和當中的'ncku.tex'
%   或'template/style/Customization.md'.

%\input{./template/style/custom/custom}

% ------------------------------------------------


% --------------------------

% 論文有關資料
% This file is need to encoded in utf-8
%
% Choose or fill in some needed information for this thesis or dissertation
% 選擇或填入你的論文一些需要使用的資料

% ----------------------------------------------------------------------------

% --- 使用的論文內容 ---
% 如果沒有打開\DemoMode
% 就會使用'./context/context.tex'中你所編寫論文內容.
% 否則會使用'./example/context.tex'的模版說明文件內容.

\DemoMode

% ----------------------------------------------------------------------------

% --- 行距 ---
% 同學可自行設定每行的距離, 這邊是以放大縮小方式來使用.
% 所以是輸入 0.1, 0.5, 1, 1.0, 1.5, 2.0, 2 等數字.
% 預設的行距: 1.2

%\SetLineStretch{1.2}

% ----------------------------------------------------------------------------

% --- 封面上語言和名字顯示方式 ---
%
% \DisplayCoverInChi:  封面以全中文顯示
% \DisplayCoverInEng:  封面以全英文顯示
% 只能選擇其中一個, 但只有最後設定的一方有效
% 預設使用\DisplayCoverInEng
%
% 另外預設在封面上只會顯示中文或英文名字而已.
% 不論你是使用\DisplayCoverInChi或\DisplayCoverInEng,
% 使用\DisplayCoverPeoplesBothNames以設定同時顯示中英文名字.

%\DisplayCoverInChi
\DisplayCoverInEng
\DisplayCoverPeoplesBothNames

% ----------------------------------------------------------------------------

% --- Title 論文題目 ---
% 填寫中文和(或)英文
% 如果題目內有必須以數學模式表示的符號,請用 \mbox{} 包住數學模式
% 如果覺得自動產生出來的題目斷行位置不適合, 可以手動加'\\'來強制斷行
% (圖書館說不管你是編寫中英混合或全英文版, 都必須同時存在中英題目)
%
% 有3種可使用, 可獨立使用, 但只有最後設定的一方有效
% \SetTitle{你的題目}{Your Title}   % 同時設定中英文題目
% \SetChiTitle{你的題目}            % 只設定中文題目
% \SetEngTitle{Your Title}         % 只設定英文題目
%
% e.g:
%
% \SetTitle %
% {國立成功大學碩博士用畢業論文LaTex模版} %
% {National Cheng Kung University (NCKU) Thesis/Dissertation Template in LaTex}
%
% or
%
% \SetChiTitle{國立成功大學碩博士用畢業論文LaTex模版}
% \SetEngTitle{National Cheng Kung University (NCKU) \\Thesis/Dissertation Template in LaTex}

\SetTitle %
{國立成功大學碩博士用畢業論文LaTex模版} %
{National Cheng Kung University (NCKU) \\Thesis/Dissertation Template in LaTex}

% ----------------------------------------------------------------------------

% --- Draft 初稿 ---
% 顯示 '(初稿)' (中文版) 和 '(Draft)' (英文版) 在封面
\DisplayDraft

% ----------------------------------------------------------------------------

% --- Degree name 學位 ---
%
% 有2種可選擇, 但只有最後設定的一方有效
% \PhdDegree    % 博士學位
% \MasterDegree % 碩士學位

\PhdDegree

% ----------------------------------------------------------------------------

% --- Your name 你的名字 ---
% 填寫你的中文和(或)英文

% 有3種可使用, 可獨立使用, 但只有最後設定的一方有效
% \SetMyName{你的名字}{Your name}   % 同時設定你的中英文名字
% \SetMyChiName{你的名字}           % 只設定你的中文名字
% \SetMyEngName{Your name}         % 只設定你的英文名字

\SetMyName{你的名字}{Your name}

% ----------------------------------------------------------------------------

% --- Date 日期 ---

% 封面日期是統一使用學位考試合格(口試合格單)單為主要參考日期 (年、月(學位考試通過日期)).
% 例如105年7月口試,則封面日期為 中華民國105年7月 或 2016年7月.

% --- 口試的日期 ---
% 設定西元的年月日, 會自動計算出民國的年份, 和英文的月份轉換
% 次序: {年份}{月份}{日}
% \SetOralDate{2016}{12}{31}

\SetOralDate{2016}{12}{31}

%--------------------------------------------------

% --- 論文封面上的日期 ---

% 如是你是國立成功大學的學生, 則封面日期直接使用口試日期, 故不需再另設定.
% 但如果你不是國立成功大學的學生, 那本模版則不清楚 貴學校所定的規範是否要分開, 故先保留這功能.

% 設定西元的年月, 會自動計算出民國的年份, 和英文的月份轉換
% 次序: {年份}{月份}
% \SetCoverDate{2016}{12}

\SetCoverDate{2016}{12}

% ----------------------------------------------------------------------------

% --- 系所 Department or Institute ---
%
% 設定你的系所名字, e.g:
% \SetDeptMath 數學系
% \SetDeptCSIE 資訊工程學系

\SetDeptCSIE

% ----------------------------------------------------------------------------

% --- 指導老師 Advisor(s) ---
% 在封面上預算了最多3位的空間
% 中文名字固定以 博士  為結尾
% 英文名字固定以 Dr. 為開頭

% 有3種可使用, 用來設定3位老師的名字
% \SetAdvisorNameX{老師的名字}{Professor's name} % 同時設定中英文名字
% \SetAdvisorChiNameX{老師的名字}                % 只設定中文名字
% \SetAdvisorEngNameX{Professor's name}         % 只設定英文名字
% (NameX為NameA, NameB, NameC)

% 使用\SetAdvisorNameA是必須的, 而如果你的指導教授有2或3位,
% 那只要增加\SetAdvisorNameB和\SetAdvisorNameC則可

\SetAdvisorNameA{A}{A}
\SetAdvisorNameB{B}{B}
\SetAdvisorNameC{C}{C}

% ----------------------------------------------------------------------------

% --- 學位考試論文證明書 Defense Certificate ---
% 使用範例版本 或 使用檔案 只能選擇其中一方

% 使用範例版本
\DisplayOralTemplate

% --- 範例版本的語言 ---
% 選擇你需要的範例
% (Only work with \DisplayOralTemplate)
% \DisplayOralChiTemplate    % 顯示中文範例版本
% \DisplayOralEngTemplate    % 顯示英文範例版本

\DisplayOralChiTemplate    % 顯示中文範例版本
\DisplayOralEngTemplate    % 顯示英文範例版本

% --- 口試委員 Committee member(s) ---
% 口試委員數量 (至少2位, 最多9位, 預設為9位)
% (Only work with \DisplayOralTemplate)
% 博士學位考試委員會置委員五人至九人
% 碩士學位考試委員會置委員三人至五人
% 口試委員人數含指導教授
\SetCommitteeSize{9}

%--------------------------------------------------

% 使用學位考試論文證明書圖片檔案
% 把你的圖片放在'context/oral'下
% 之後設定中英文版所對應是哪一個檔案
% 就算已啟用\DisplayOralImage,
% 但沒有填寫圖檔檔名的話, 都不會顯示出來.
% (例子用的'example-oral-chi.pdf'和'example-oral-eng.pdf'已放在'context/oral'中)

%\DisplayOralImage                % 顯示圖檔
%\SetOralImageChi{example-oral-chi.pdf}   % 中文版檔案
%\SetOralImageEng{example-oral-eng.pdf}   % 英文版檔案

% ----------------------------------------------------------------------------

% --- 關鍵字 Keyword ---
% 最多9個關鍵字
% 為了方便同學自行設定
% 故所產出來的PDF檔案中的關鍵字和內文摘要的關鍵字
% 可獨立個別設定

% \SetKeywords是設定所產出來的PDF中的Keyword項目
% 可同時填寫中英文
% e.g
% \SetKeywords{Keyword A (關鍵字 A)}{Keyword B (關鍵字 B)}{Keyword C (關鍵字 C)}
% 或單純中文或英文
% \SetKeywords{Keyword A}{Keyword B}{Keyword C}
% \SetKeywords{關鍵字 A}{關鍵字 B}{關鍵字 C}

\SetKeywords{NCKU Thesis/Dissertation template}{Graduate}{LaTex/XeLaTex}

% 摘要中的關鍵字
% 為了方便同學們能達到以下情況:
% a. 只寫中文版摘要
% b. 只寫英文版摘要
% c. 同時寫中英文版摘要
% 故中英文版的關鍵字都是可個別設定
% \SetAbstractChiKeywords: 用來設定中文版摘要的關鍵字
% \SetAbstractEngKeywords: 用來設定英文版摘要的關鍵字
% \SetAbstractExtKeywords: 用來設定英文延伸摘要的關鍵字 (只有你要編寫英文延伸摘要才需要設定)
% 所以只要使用你需要寫的版本則可.
% 但如果2個版本都要寫, 則2個都同時使用則可.
% 沒有填寫的話, 則摘要中的關鍵字部份是不會顯示出來.
%
% e.g
% \SetAbstractChiKeywords{關鍵字 A}{關鍵字 B}{關鍵字 C}
% \SetAbstractEngKeywords{Keyword A}{Keyword B}{Keyword C}
% \SetAbstractExtKeywords{Keyword A}{Keyword B}{Keyword C}
% 英文延伸摘要的關鍵字理應會跟英文版摘要的關鍵字是一樣,
% 但為了同學能編寫不同內容和關鍵字, 故可獨立設定.

\SetAbstractChiKeywords{國立成功大學畢業論文模版}{碩博士}{LaTex/XeLaTex}
\SetAbstractEngKeywords{NCKU Thesis/Dissertation Template}{Graduate}{LaTex/XeLaTex}
\SetAbstractExtKeywords{NCKU Thesis/Dissertation Template}{Graduate}{LaTex/XeLaTex}

% ----------------------------------------------------------------------------

% --- 目錄 Index ---
% 設定可獨立使用, 但只有最後設定的一方有效

% 標題文字語言 Language
% 目錄的標題文字使用預設的中文或是英文
% \IndexChiMode:  標題文字為中文
% \IndexEngMode:  標題文字為英文
% 預設的目錄標題為: 目錄 (中文) / Table of Contents (英文)
% 預設的表格目錄標題為: 表格 (中文) / List of Tables (英文)
% 預設的圖片目錄標題為: 圖片 (中文) / List of Figures (英文)
% 預設使用\IndexEngMode

%\IndexChiMode
\IndexEngMode

% ----------------------

% 目錄標題文字 Text of title
% 如果預設文字不是你所希望的, 那可以使用這邊去個別設定你所希望的文字, 不分中英文.

% 設定目錄標題
%\SetIndexTitleText{Table of Contents / 目錄}

% 設定表格目錄標題
%\SetTablesIndexTitleText{List of Tables / 表格}

% 設定圖片目錄標題
%\SetFiguresIndexTitleText{List of Figures / 圖片}

% ----------------------------------------------------------------------------

% --- 圖片相關的設定 ---
% 預設上每一張圖的名字都是以 'Figure 2.1'
% 假如想使用自定的名字, 如 '圖 2.1'
% 則使用 \SetCustomFigureName{圖} 即可.

%\SetCustomFigureName{Figure}

% ----------------------------------------------------------------------------

% --- 表格相關的設定 ---
% 預設上每一張表的名字都是以 'Table 2.1'
% 假如想使用自定的名字, 如 '表 2.1'
% 則使用 \SetCustomTableName{表} 即可.

%\SetCustomTableName{Table}

% ----------------------------------------------------------------------------

% --- 參考文獻 Reference ---
%
% --- 使用方式 ---
%    \SetupReference{ < 設定 > }
%
% < 設定 >
%    Title (Reference的標題文字)
%    BibStyle (Reference引用時的格式)
%
% ----------------------
%
% < Title >
% 模版提供了一些預設的文字
%    \TextDefaultTitleReferenceChi: 參考文獻
%    \TextDefaultTitleReferenceEng: References
%    \TextDefaultTitleBibliographyEng: Bibliography
%
% 預設上是使用\TextDefaultTitleReferenceEng.
%
% 使用時:
%    Title={\TextDefaultTitleBibliographyEng}
%
% 或自定你的文字:
%    Title={我的參考文獻標題}
%
% ----------------------
%
% < BibStyle >
% 除非有特殊的格式要求, 否則這部份是不用管的.
%
% 使用的格式 | 作者名稱顯示的格式 | 引用時顯示的例子
%   abbrv | H. J. Simpson | [4]
%   plain | Homer Jay Simpson | [4]
%   alpha | Homer Jay Simpson | Sim95
%   apacite | Homer J. S. | Homer, 1995
%
% 模版提供了:
%   LaTex基本格式:
%       abbrv, acm, alpha, apalike, ieeetr, plain, siam, unsrt
%   可參考<https://www.sharelatex.com/learn/Bibtex_bibliography_styles>
%
%   額外的格式:
%      apacite
%   可參考<https://www.overleaf.com/latex/examples/package-example-apacite/jkssmfcpzwmy>
%
% 預設使用plain.
%
% 注意:
%   如果你要轉換使用格式時, 推薦在重新產生論文前, 先把所有除了thesis.tex外的所有
%   thesis開頭或以thesis為檔名的檔案全刪掉.
%   例如'thesis.bbl', 'thesis.aux', 'thesis.lof'等所有檔案.
%   否則有可能在產生論文時遇到錯誤, 如果遇到錯誤, 請不斷重新刪掉和重新產生論文,
%   直到解決問題為止.
%
% 已知:
%   由abbrv/plain轉去apacite必定需要刪除檔案才能進行.
% ----------------------
%
% --- 請在這邊設定你要的樣子 ---
%

%\SetupReference{%
%  Title = {\TextDefaultTitleReferenceEng},
%  BibStyle = {plain},
%} % End of \SetupReference{}

% ----------------------------------------------------------------------------

% --- 章節標題的設定 ---
% 除非對章節標題格式有任何要求, 否則這部份內容是不用管的.
%
% 模版的章節有一個預設的格式:
%
% 一般章節:
%    Chapter: Chapter 1
%    Section: 1.1
%    SubSection: 1.1.1
%    SubSubSection: (空白, 只有題目)
%
% 附錄章節:
%    Chapter: Appendix A
%    Section: A.1
%    SubSection: A.1.1
%    SubSubSection: (空白, 只有題目)
%
% 如對格式有什麼的要求, 請使用\SetNumberingFormat.

% --- 使用方式 ---
%    \SetNumberingFormat[ < 章節類型 > ]{ < 設定 > }

% ----------------------
%
% < 章節類型 >
% 針對每一種的章節都可自設自己需要的格式,
% 有8種類型提供, 包括一般章節和附錄章節.
%    Chapter (章)
%    Section (節)
%    SubSection (小節)
%    SubSubSection (小小節)
%    AppendixChapter (附錄中的章)
%    AppendixSection (附錄中的節)
%    AppendixSubSection (附錄中的小節)
%    AppendixSubSubSection (附錄中的小小節)
%
% ----------------------
%
% < 設定 >
% 以下的設定針對標題中不同內容的設定.
%    BeginText (章節號碼前面的文字)
%    EndText (章節號碼後面的文字)
%    TextAlign (標題文字的位置)
%    CNumStyle ('章' 的數字類型)
%    SNumStyle ('節' 的數字類型)
%    SSNumStyle ('小節' 的數字類型)
%    SSSNumStyle ('小小節' 的數字類型)
%    SepAtIndex (目錄中章節號碼跟章節題目中的分隔符號)
%    SepBetweenCnS ('章' 號碼跟 '節' 號碼中間的分隔符號)
%    SepBetweenSnSS ('節' 號碼跟 '小節' 號碼中間的分隔符號)
%    SepBetweenSSCnSSS ('小節' 號碼跟 '小小節' 號碼中間的分隔符號)
%
% 標題在不同位置使用的內容都不一樣:
%
% 內文:
%   <BeginText> @NUMBER@ <EndText>
%   例如: 第2章
%
%   @NUMBER@為第幾章節的那個數字
%       <CNumStyle> <SepBetweenCnS>
%         <SNumStyle> <SepBetweenSnSS>
%           <SSNumStyle> <SepBetweenSSCnSSS> <SSSNumStyle>
%   例如: 2.1, 3.1.2, A.2
%
% 目錄:
%   <BeginText> @NUMBER@ <EndText> <SepAtIndex> @TITLE@
%   例如: 第2章. 介紹
%
% 被引用時:
%   @NUMBER@
%   例如: 2.1, 3.1.2, A.2
%
% ----------------------
%
% 一個完整的 \SetNumberingFormat 的樣子:
% \SetNumberingFormat[ < 章節類型 > ]{%
%   BeginText = { @文字/符號@ }, EndText = { @文字/符號@ },%
%   TextAlign = { @Left/Center/Right@ },%
%   CNumStyle = { < 數字類型 > }, SNumStyle = { < 數字類型 > },%
%   SSNumStyle = { < 數字類型 > }, SSSNumStyle = { < 數字類型 > },%
%   SepAtIndex = { @文字/符號@ }, SepBetweenCnS = { @文字/符號@ },%
%   SepBetweenSnSS = { @文字/符號@ }, SepBetweenSSCnSSS = { @文字/符號@ },%
% } % End of \SetNumberingFormat{}
%
% ----------------------
%
% --- 數字類型 ---
%
% 模版提供以下的數字類型使用
%    ChiNum (使用 '中文數字' 方式, 如: 一二三)
%    Tiangan 使用 '天干' 方式, 如: 甲乙丙丁戊癸)
%    Arabic (使用 '阿拉伯數字' 方式, 如: 1 2 3 4 5 6)
%    LowerRoman (使用 '小寫羅馬數字' 方式, 如: i ii iii vi x)
%    UpperRoman (使用 '大寫羅馬數字' 方式, 如: I II III VI X)
%    LowerAlph (使用 '小寫英文字母' 方式, 如: a b c)
%    UpperAlph (使用 '大寫英文字母' 方式, 如: A B C)
%
% 選擇你想要的數字類型後, 在<設定>中的這些位置填寫你要的類型
%    CNumStyle
%    SNumStyle
%    SSNumStyle
%    SSSNumStyle
%
%   例如: CNumStyle={Arabic}
%
% ----------------------
%
% --- 標題文字位置 ---
%
% 模版提供以下的位置使用
%   Left: 左邊
%   Center: 置中
%   Right: 右邊
% 預設上所有章節都是Left.
%
% 例如: TextAlign={Center}
%
% ----------------------
%
% --- 例子 ---
%
% 如果 '章' 要由文字改使用為:
%        'Chapter 1' -> '第1章'
% 則使用
%   \SetNumberingFormat[Chapter]{%
%     BeginText = {第}, EndText = {章}%
%   }%
%
% -----------
%
% 如果 '附錄的章' 要由文字改使用為:
%        'Appendix A' -> '附錄 A'
% 則使用
%   \SetNumberingFormat[AppendixChapter]{%
%     BeginText = {附錄 }%
%   }%
%
% -----------
%
% 如果 '章' 要由數字改使用為:
%        'Chapter 1' -> 'Chapter -A-'
% 則使用
%   \SetNumberingFormat[Chapter]{%
%     BeginText = {Chapter -}, EndText = {-},%
%     CNumStyle = {UpperAlph},%
%    }%
%
% -----------
%
% 如果 '節' 要由數字改使用為:
%        '1.2' -> '一 -乙-'
% 則使用
%   \SetNumberingFormat[Section]{%
%     EndText = {-},%
%     CNumStyle = {ChiNum}, SNumStyle = {Tiangan},%
%     SepBetweenCnS = { -},%
%    }%
%
% -----------
%
% 如果 '節' 不想看到 '章' 的數字:
%        '1.2' -> '(2)'
% 則使用
%   \SetNumberingFormat[Section]{%
%     BeginText = {(}, EndText = {)},%
%     CNumStyle = {}, SNumStyle = {Arabic},%
%     SepBetweenCnS = {},%
%    }%
% 不提供 '章' 的數字類型跟中間的分隔符號
%
% ----------------------
%
% 目錄中章節號碼跟章節題目中的分隔符號
% 正常在目錄中會顯示 'Chapter 1. ABCDEF' 或 '第一章. ABCDEF'
% 但因個人喜好, 做法不一樣, 如 'Chapter 1: ABCDEF' 或 '第一章 ABCDEF'
% 故使用 SepAtIndex 可設定你想要的符號或不需要符號
%
% 如想換'章'的由'Chapter 1. ABCDEF'換成'Chapter 1: ABCDEF'
% 則使用
%   \SetNumberingFormat[Chapter]{%
%     SepAtIndex = {:},%
%    }%
%
% 如想換'章'的由'第一章. ABCDEF'換成'第一章 ABCDEF'
%   \SetNumberingFormat[Chapter]{%
%     BeginText = {第}, EndText = {章},%
%     CNumStyle = {ChiNum},%
%     SepAtIndex = {},%
%   }%
% ----------------------
%
% --- 請在這邊設定你要的樣子 ---
%

% Chapter (章)
%\SetNumberingFormat[Chapter]{%
%  BeginText = {Chapter }, EndText = {},
%  CNumStyle = {Arabic},
%  SepAtIndex = {.},
%} % End of \SetNumberingFormat{}

% Section (節)
\SetNumberingFormat[Section]{%
  BeginText = {}, EndText = {},
  TextAlign = {Left},
  CNumStyle = {}, SNumStyle = {},
  SepAtIndex = {}, SepBetweenCnS = {},
} % End of \SetNumberingFormat{}

% SubSection (小節)
\SetNumberingFormat[SubSection]{%
  BeginText = {}, EndText = {},
  TextAlign = {Left},
  CNumStyle = {}, SNumStyle = {}, SSNumStyle = {},
  SepAtIndex = {.}, SepBetweenCnS = {}, SepBetweenSnSS = {},
} % End of \SetNumberingFormat{}

% SubSubSection (小小節)
%\SetNumberingFormat[SubSubSection]{%
%  BeginText = {}, EndText = {},
%  TextAlign = {Left},
%  CNumStyle = {}, SNumStyle = {}, SSNumStyle = {}, SSSNumStyle = {},
%  SepAtIndex = {}, SepBetweenCnS = {},
%  SepBetweenSnSS = {}, SepBetweenSSCnSSS = {},
%} % End of \SetNumberingFormat{}

% AppendixChapter (附錄中的章)
%\SetNumberingFormat[AppendixChapter]{%
%  BeginText = {Appendix }, EndText = {},
%  CNumStyle = {UpperAlph},
%  SepAtIndex = {.},
%} % End of \SetNumberingFormat{}

% AppendixSection (附錄中的節)
%\SetNumberingFormat[AppendixSection]{%
%  BeginText = {}, EndText = {},
%  TextAlign = {Left},
%  CNumStyle = {UpperAlph}, SNumStyle = {Arabic},
%  SepAtIndex = {.}, SepBetweenCnS = {.},
%} % End of \SetNumberingFormat{}

% AppendixSubSection (附錄中的小節)
%\SetNumberingFormat[AppendixSubSection]{%
%  BeginText = {}, EndText = {},
%  TextAlign = {Left},
%  CNumStyle = {UpperAlph}, SNumStyle = {Arabic}, SSNumStyle = {Arabic},
%  SepAtIndex = {.}, SepBetweenCnS = {.}, SepBetweenSnSS = {.},
%} % End of \SetNumberingFormat{}

% AppendixSubSubSection (附錄中的小小節)
%\SetNumberingFormat[AppendixSubSubSection]{%
%  BeginText = {}, EndText = {},
%  TextAlign = {Left},
%  CNumStyle = {}, SNumStyle = {}, SSNumStyle = {}, SSSNumStyle = {},
%  SepAtIndex = {}, SepBetweenCnS = {},
%  SepBetweenSnSS = {}, SepBetweenSSCnSSS = {},
%} % End of \SetNumberingFormat{}

% ----------------------------------------------------------------------------

% --- Theorems的設定 ---
% 除非對Theorems格式有任何要求, 否則這部份內容是不用管的.
%
% 提供以下的Theorems的使用:
%
%    Definition       (定義)
%    Condition        (條件)
%    Theorem          (定理)
%    Lemma            (引理)
%    Example          (例子)
%    Corollary        (推論)
%    Proposition      (主張)
%    Proof            (證明)
%    Conjecture       (猜想)
%    Note             (附註)
%    Annotation       (註解)
%    Claim            (主張)
%    Case             (情況)
%    Acknowledgment   (確認)
%    Conclusion       (結論)
%    Criterion        (標準)
%    Assertion        (斷言)
%    Problem          (問題)
%    Question         (問題)
%    Hypothesis       (假設)
%    Summary          (總結)
%
% 如對格式有什麼的要求, 請使用\SetNumberingFormat.
% 而插入新內容的話則使用\InsertXXXX.
%
% --- 使用方式 ---
%   \SetTheoremFormat[ < Theorem類型 > ]{ < 設定 > }
%     和
%   \InsertXXXX[ < 設定 > ]{ < 內容 >} (XXXX為Theorem類型), 如
%     \InsertTheorem{ abc }
%     \InsertLemma{ abc }
%     \InsertProof{ abc }
%
% ----------------------
%
% 一個完整的 \SetTheoremFormat 的樣子:
%
% \SetTheoremFormat[ < Theorem類型 > ]{%
%   ShowText = { @文字/符號@ },
%   FollowCounter = { < Counter類型 > },%
% } % End of \SetTheoremFormat{}
%
% ----------------------
%
% --- ShowText ---
%
% ShowText是指所顯示在文章中的文字. 如Proof可修改成:
%    ShowText = {證明}
%
% ----------------------
%
% --- Counter類型 ---
%
% 模版提供以下的Counter類型使用:
%
%    Section
%    Definition
%    Condition
%    Theorem
%    Lemma
%    Example
%    Corollary
%    Proposition
%    Proof
%    Conjecture
%    Note
%    Annotation
%    Claim
%    Case
%    Acknowledgment
%    Conclusion
%    Criterion
%    Assertion
%    Problem
%    Question
%    Hypothesis
%    Summary
%
%  有一些Theorem是不需要Counter, 故那些是不會需要這設定. 如Proof/Note.
%  而需要的則全預設跟隨Section Counter.
%
%  以下為預設使用Counter的清單:
%
%    Definition
%    Condition
%    Theorem
%    Lemma
%    Example
%    Corollary
%    Proposition
%    Conjecture
%    Criterion
%    Assertion
%    Problem
%    Question
%    Hypothesis
%
% ----------------------
%
% --- 請在這邊設定你要的樣子 ---
%

%\SetTheoremFormat[Definition]{ShowText = {Definition}}%
%\SetTheoremFormat[Condition]{ShowText = {Condition}}%
%\SetTheoremFormat[Problem]{ShowText = {Problem}}%
%\SetTheoremFormat[Example]{ShowText = {Example}}%
%\SetTheoremFormat[Theorem]{ShowText = {Theorem}}%
%\SetTheoremFormat[Lemma]{ShowText = {Lemma}}%
%\SetTheoremFormat[Corollary]{ShowText = {Corollary}}%
%\SetTheoremFormat[Proposition]{ShowText = {Proposition}}%
%\SetTheoremFormat[Conjecture]{ShowText = {Conjecture}}%
%\SetTheoremFormat[Proof]{ShowText = {Proof}}%
%\SetTheoremFormat[Note]{ShowText = {Note}}%
%\SetTheoremFormat[Annotation]{ShowText = {Annotation}}%
%\SetTheoremFormat[Claim]{ShowText = {Claim}}%
%\SetTheoremFormat[Case]{ShowText = {Case}}%
%\SetTheoremFormat[Acknowledgment]{ShowText = {Acknowledgment}}%
%\SetTheoremFormat[Conclusion]{ShowText = {Conclusion}}%
%\SetTheoremFormat[Criterion]{ShowText = {Criterion}}%
%\SetTheoremFormat[Assertion]{ShowText = {Assertion}}%
%\SetTheoremFormat[Question]{ShowText = {Question}}%
%\SetTheoremFormat[Hypothesis]{ShowText = {Hypothesis}}%
%\SetTheoremFormat[Summary]{ShowText = {Summary}}%

% ----------------------------------------------------------------------------


% --------------------------

% 在 pdf 簡介欄裡填入相關資料
\FillInPDFData

% ------------------------------------------------

% 一些會受到conf.tex中設定而影響的package或排版的設定

% Makes all pages the height of the text on that page.
% No extra vertical space is added.
%\raggedbottom

% Setup all custom numbering format
\SetupNumberingFormat

% 當所有的package都include完後, 才真正設定我們要的字型,
% 以清掉所有由package影響到的設定.
\InitDefaultFontType

%\setlength{\parindent}{4em}
%\usepackage{indentfirst}

% Initinal all theorem formats
\InitTheoremFormats

% ------------------------------------------------


% 產生封面
%
% 封面: 顯示所有封面內容, 但沒有學校Logo)
%     主要用在印刷版, 如精裝版 或 平裝版
%
% 內頁: 顯示所有封面內容, 但有學校Logo
%     主要用在電子版 + 印刷版
%     (在context.tex設定)
%
% 只要是印刷版, 不論是精裝版或平裝版, 都是 封面 (殼/皮) + 內頁.
% 只有在電子版時, 第一頁就是封面內頁.
%
% 提供封面的PDF給影印店時,
% 記得不要把檔案上鎖,
% 免得增加影印店去做外皮時的不便.
\CreateCover

  \end{document}
} % End of \newcommand{}

% Use to include and display inner cover
\newcommand{\DisplayInnerCover}{%
% This file is part of the project of
% National Cheng Kung University (NCKU) Thesis/Dissertation Template in LaTex.
% This project is hold at
%     <https://github.com/wengan-li/ncku-thesis-template-latex>
% by Wen-Gan Li.
%
% This project is distributed in the hope of usefuling to someone,
% you can redistribute it and/or modify it under the terms of the
% Attribution-NonCommercial-ShareAlike 4.0 International.
%
% You should have received a copy of the
% Attribution-NonCommercial-ShareAlike 4.0 International
% along with this project.
% If not, see <http://creativecommons.org/licenses/by-nc-sa/4.0/legalcode.txt>.
%
% Please feel free to fork it, modify it, and try it.
% Have fun !!!
%

% ------------------------------------------------

% 根據user的需求去選用中文或英文封面內頁
%
% 封面: 顯示所有封面內容, 但沒有學校Logo
% 內頁: 顯示所有封面內容, 但有學校Logo
% 不論是精裝版或平裝版都是 封面 (殼/皮) + 內頁
%
\if \GetDisplayCoverLang \ValueDisplayCoverLangEng
  %
% This file is part of the project of
% National Cheng Kung University (NCKU) Thesis/Dissertation Template in LaTex.
% This project is hold at
%     <https://github.com/wengan-li/ncku-thesis-template-latex>
% by Wen-Gan Li.
%
% This project is distributed in the hope of usefuling to someone,
% you can redistribute it and/or modify it under the terms of the
% Attribution-NonCommercial-ShareAlike 4.0 International.
%
% You should have received a copy of the
% Attribution-NonCommercial-ShareAlike 4.0 International
% along with this project.
% If not, see <http://creativecommons.org/licenses/by-nc-sa/4.0/legalcode.txt>.
%
% Please feel free to fork it, modify it, and try it.
% Have fun !!!
%

% ----------------------------------------------------------------------------
%                English cover
%                   英文封面
% ----------------------------------------------------------------------------

% ------------------------------------------------
\StartCover
% ------------------------------------------------

% 顯示 校名, 系所名, 論文種類
\begin{minipage}[c][5cm][t]{\textwidth}
  \begin{center}
    \vspace{0.8cm}
    \makebox[\textwidth][c]{\Huge \GetUniversityEngName} \\

    \vspace{0.5cm}
    \makebox[\textwidth][c]{\LARGE \GetDeptEngName} \\

    \vspace{0.5cm}
    \makebox[\textwidth][c]{\LARGE \GetEngDegreeThesis} \\

    \ifthenelse{\equal{\GetFlagDisplayDraft}{1}}%
      {\vspace{0.5cm}\makebox[5cm][c]{\LARGE \GetTextDraftEng}}{}
  \end{center}
\end{minipage}

% ------------------------------------------------

\vspace{5.5cm}

% ------------------------------------------------

% English title 英文題目
\begin{minipage}[c][5cm][t]{\textwidth}
  \begin{center}
%    \parbox{\textwidth}{\center \LARGE \GetEngTitle}
%    \parbox{\textwidth}{\center \Large \GetChiTitle}
    \makebox[\textwidth][c]{\parbox{\paperwidth }{\center \Large \GetChiTitle}}

    \vspace{0.5cm}

%    \parbox{\textwidth}{\center \Large \GetEngTitle}
    \makebox[\textwidth][c]{\parbox{\paperwidth }{\center \Large \GetEngTitle}}
  \end{center}
\end{minipage}

% ------------------------------------------------

\vspace{1.0cm}

% ------------------------------------------------

% 顯示學生和老師的名字

\begin{minipage}[c][4.5cm][t]{\textwidth}
  \begin{center}
    \ifthenelse{\equal{\GetCDBothName}{1}}%
      {%
        % ----- 中英文同時顯示 -----
        % --------------------------
        % 顯示 學生 的名字
        \hspace{2.0em}
        \makebox[4.0em][r]{\Large 學生:}
        \makebox[6.0em][l]{\Large \GetAuthorChiName}
        \makebox[8.0em][c]{}
        \makebox[4.0em][r]{\Large Student:}
        \makebox[8.0em][l]{\Large \GetAuthorEngName}\\
        % --------------------------
        \vspace{0.5cm}
        % --------------------------
        % 顯示 指導老師 A 的名字
        \hspace{2.0em}
        \makebox[4.0em][r]{\Large 指導老師:}
        \makebox[6.0em][l]{\Large \GetAdvisorChiNameA \thinspace 博士}
        \makebox[8.0em][c]{}
        \makebox[4.0em][r]{\Large Advisor:}
        \makebox[8.0em][l]{\Large Dr. \thinspace \GetAdvisorEngNameA}\\
        % --------------------------
        \vspace{0.1cm}
        % --------------------------
        % 顯示 指導老師 B 的名字
        \hspace{2.0em}
        \makebox[4.0em][r]
        {%
          \ifthenelse{\equal{\GetAdvisorChiNameB}{\empty}}%
            {}%
            {\Large 共同指導:}%
        }
        \makebox[6.0em][l]%
        {%
          \ifthenelse{\equal{\GetAdvisorChiNameB}{\empty}}%
            {}%
            {\Large \GetAdvisorChiNameB \thinspace 博士}%
        }
        \makebox[8.0em][c]{}
        \makebox[4.0em][r]
        {%
          \ifthenelse{\equal{\GetAdvisorEngNameB}{\empty}}%
            {}%
            {\Large Co-Advisor:}%
        }
        \makebox[8.0em][l]%
        {%
          \ifthenelse{\equal{\GetAdvisorEngNameB}{\empty}}%
            {}%
            {\Large Dr. \thinspace \GetAdvisorEngNameB}%
        } \\
        % --------------------------
        \vspace{0.1cm}
        % --------------------------
        % 顯示 指導老師 C 的名字
        \hspace{2.0em}
        \makebox[4.0em][r]{}
        \makebox[6.0em][l]
        {%
          \ifthenelse{\equal{\GetAdvisorChiNameC}{\empty}}%
            {}%
            {\Large \GetAdvisorChiNameC \thinspace 博士}%
        }
        \makebox[8.0em][c]{}
        \makebox[4.0em][r]{}
        \makebox[8.0em][l]
        {%
          \ifthenelse{\equal{\GetAdvisorEngNameC}{\empty}}%
            {}%
            {\Large Dr. \thinspace \GetAdvisorEngNameC}%
        } \\
        % --------------------------
      }%
      {%
        % ----- 只顯示英文 -----
        % --------------------------
        % 顯示 學生 的名字
        \makebox[7em][r]{\Large Student:}
        \makebox[10.0em][l]{\Large \GetAuthorEngName}\\
        % --------------------------
        \vspace{0.5cm}
        % --------------------------
        % 顯示 指導老師 A 的名字
        \makebox[7em][r]{\Large Advisor:}
        \makebox[10.0em][l]{\Large Dr. \thinspace \GetAdvisorEngNameA} \\
        % --------------------------
        \vspace{0.1cm}
        % --------------------------
        % 顯示 指導老師 B 的名字
        \makebox[7em][r]
        {%
          \ifthenelse{\equal{\GetAdvisorEngNameB}{\empty}}%
            {}%
            {\Large Co-Advisor:}%
        }
        \makebox[10.0em][l]
        {%
          \ifthenelse{\equal{\GetAdvisorEngNameB}{\empty}}%
            {}%
            {\Large Dr. \thinspace \GetAdvisorEngNameB}%
        } \\
        % --------------------------
        \vspace{0.1cm}
        % --------------------------
        % 顯示 指導老師 C 的名字
        \makebox[7em][r]{}
        \makebox[10.0em][l]
        {%
          \ifthenelse{\equal{\GetAdvisorEngNameC}{\empty}}%
            {}%
            {\Large Dr. \thinspace \GetAdvisorEngNameC}%
        } \\
        % --------------------------
      }%
  \end{center}
\end{minipage}
% ------------------------------------------------

% Date 日期
\vspace{0.5cm}
\begin{minipage}{\textwidth}
  \begin{center}\Large %
    \ifthenelse{\equal{\GetFlagDegreeType}{\ValueDegreeMaster}}%
      {\GetThesisMonthInEng \thinspace \GetThesisYear}%
      {\GetOralEngDay \thinspace \thinspace \GetThesisMonthInEng \thinspace \thinspace \GetThesisYear}
  \end{center}
\end{minipage}

% ------------------------------------------------
\EndCover
% ------------------------------------------------

\else
  %
% This file is part of the project of
% National Cheng Kung University (NCKU) Thesis/Dissertation Template in LaTex.
% This project is hold at
%     <https://github.com/wengan-li/ncku-thesis-template-latex>
% by Wen-Gan Li.
%
% This project is distributed in the hope of usefuling to someone,
% you can redistribute it and/or modify it under the terms of the
% Attribution-NonCommercial-ShareAlike 4.0 International.
%
% You should have received a copy of the
% Attribution-NonCommercial-ShareAlike 4.0 International
% along with this project.
% If not, see <http://creativecommons.org/licenses/by-nc-sa/4.0/legalcode.txt>.
%
% Please feel free to fork it, modify it, and try it.
% Have fun !!!
%

% ----------------------------------------------------------------------------
%                Chinese cover
%                   中文封面
% ----------------------------------------------------------------------------

% ------------------------------------------------
\StartCover
% ------------------------------------------------

% 顯示 校名, 系所名, 論文種類
\begin{minipage}[c][5cm][t]{\textwidth}
  \begin{center}
    \makebox[10cm][s]{\Huge \GetUniversityChiName} \\

    \vspace{1cm}

    \makebox[8cm][s]{\Huge \GetDeptChiName} \\

    \vspace{1cm}

    \makebox[5cm][s]{\Huge \GetChiDegree 論文}%

    \ifthenelse{\equal{\GetFlagDisplayDraft}{1}}%
      {\vspace{1cm}\makebox[5cm][c]{\Huge \GetTextDraftChi}}{}
  \end{center}
\end{minipage}
% ------------------------------------------------

\vspace{5.5cm}

% ------------------------------------------------

% Chinese and English title 中英文題目
\begin{minipage}[c][5cm][t]{\textwidth}
  \begin{center}
    \makebox[\textwidth][c]{\parbox{\paperwidth}{\center \Large \GetChiTitle}}

    \vspace{0.5cm}
%    \parbox{\textwidth}{\center \Large \GetEngTitle}
    \makebox[\textwidth][c]{\parbox{\paperwidth}{\center \Large \GetEngTitle}}
  \end{center}
\end{minipage}

% ------------------------------------------------

\vspace{1.0cm}

% ------------------------------------------------

% 顯示學生和老師的名字

\begin{minipage}[c][4.5cm][t]{\textwidth}
  \begin{center}
    \ifthenelse{\equal{\GetCDBothName}{1}}%
      {%
        % ----- 中英文同時顯示 -----
        % --------------------------
        % 顯示 學生 的名字
        \hspace{2.0em}
        \makebox[4.0em][r]{\Large 學生:}
        \makebox[6.0em][l]{\Large \GetAuthorChiName}
        \makebox[8.0em][c]{}
        \makebox[4.0em][r]{\Large Student:}
        \makebox[8.0em][l]{\Large \GetAuthorEngName}\\
        % --------------------------
        \vspace{0.5cm}
        % --------------------------
        % 顯示 指導老師 A 的名字
        \hspace{2.0em}
        \makebox[4.0em][r]{\Large 指導老師:}
        \makebox[6.0em][l]{\Large \GetAdvisorChiNameA \thinspace 博士}
        \makebox[8.0em][c]{}
        \makebox[4.0em][r]{\Large Advisor:}
        \makebox[8.0em][l]{\Large Dr. \thinspace \GetAdvisorEngNameA}\\
        % --------------------------
        \vspace{0.1cm}
        % --------------------------
        % 顯示 指導老師 B 的名字
        \hspace{2.0em}
        \makebox[4.0em][r]
        {%
          \ifthenelse{\equal{\GetAdvisorChiNameB}{\empty}}%
            {}%
            {\Large 共同指導:}%
        }
        \makebox[6.0em][l]%
        {%
          \ifthenelse{\equal{\GetAdvisorChiNameB}{\empty}}%
            {}%
            {\Large \GetAdvisorChiNameB \thinspace 博士}%
        }
        \makebox[8.0em][c]{}
        \makebox[4.0em][r]
        {%
          \ifthenelse{\equal{\GetAdvisorEngNameB}{\empty}}%
            {}%
            {\Large Co-Advisor:}%
        }
        \makebox[8.0em][l]%
        {%
          \ifthenelse{\equal{\GetAdvisorEngNameB}{\empty}}%
            {}%
            {\Large Dr. \thinspace \GetAdvisorEngNameB}%
        } \\
        % --------------------------
        \vspace{0.1cm}
        % --------------------------
        % 顯示 指導老師 C 的名字
        \hspace{2.0em}
        \makebox[4.0em][r]{}
        \makebox[6.0em][l]
        {%
          \ifthenelse{\equal{\GetAdvisorChiNameC}{\empty}}%
            {}%
            {\Large \GetAdvisorChiNameC \thinspace 博士}%
        }
        \makebox[8.0em][c]{}
        \makebox[4.0em][r]{}
        \makebox[8.0em][l]
        {%
          \ifthenelse{\equal{\GetAdvisorEngNameC}{\empty}}%
            {}%
            {\Large Dr. \thinspace \GetAdvisorEngNameC}%
        } \\
        % --------------------------
      }%
      {%
        % ----- 只顯示中文 -----
        % --------------------------
        % 顯示 學生 的名字
        \makebox[7em][r]{\Large 研究生:}
        \makebox[10.0em][l]{\Large \GetAuthorChiName}\\
        % --------------------------
        \vspace{0.5cm}
        % --------------------------
        % 顯示 指導老師 A 的名字
        \makebox[7em][r]{\Large 指導老師:}
        \makebox[10.0em][l]{\Large \GetAdvisorChiNameA \thinspace 博士} \\
        % --------------------------
        \vspace{0.1cm}
        % --------------------------
        % 顯示 指導老師 B 的名字
        \makebox[7em][r]
        {%
          \ifthenelse{\equal{\GetAdvisorChiNameB}{\empty}}%
            {}%
            {\Large 共同指導:}%
        }
        \makebox[10.0em][l]
        {%
          \ifthenelse{\equal{\GetAdvisorChiNameB}{\empty}}%
            {}%
            {\Large \GetAdvisorChiNameB \thinspace 博士}%
        } \\
        % --------------------------
        \vspace{0.1cm}
        % --------------------------
        % 顯示 指導老師 C 的名字
        \makebox[7em][r]{}
        \makebox[10.0em][l]
        {%
          \ifthenelse{\equal{\GetAdvisorChiNameC}{\empty}}%
            {}%
            {\Large \GetAdvisorChiNameC \thinspace 博士}%
        } \\
        % --------------------------
      }%
  \end{center}
\end{minipage}
% ------------------------------------------------

% Date 日期
\vspace{0.5cm}
\ifthenelse{\equal{\GetFlagDegreeType}{\ValueDegreeMaster}}%
{%
  \ifthenelse{\equal{\VarDisplayCoverDateNum}{\ValueDisplayCoverDateNumInChi}}%
  {\makebox[8cm][s]{\Large 中華民國\zhnumber{\GetThesisYearInTaiwanYear}年\zhnumber{\GetThesisMonth}月}}%
  {\makebox[8cm][s]{\Large 中華民國 \GetThesisYearInTaiwanYear 年 \GetThesisMonth 月}}%
}%
{
  \ifthenelse{\equal{\VarDisplayCoverDateNum}{\ValueDisplayCoverDateNumInChi}}%
  {%
    \makebox[8cm][s]{\Large 中華民國 \GetOralYearInTaiwanYearNumInChi 年 \GetThesisMonthNumInChi 月}%
    %\makebox[8cm][s]{\Large 中華民國 \GetOralYearInTaiwanYearNumInChi 年 \GetThesisMonthNumInChi 月 \GetOralEngDayNumInChi 日}%
  }%
  {%
    \makebox[8cm][s]{\Large 中華民國 \GetOralYearInTaiwanYear 年 \GetThesisMonth 月}%
    %\makebox[8cm][s]{\Large 中華民國 \GetOralYearInTaiwanYear 年 \GetThesisMonth 月 \GetOralEngDay 日}%
  }%
}

% ------------------------------------------------
\EndCover
% ------------------------------------------------

\fi

% ------------------------------------------------
}

% ----------------------------------------------------------------------------

\newcommand{\ValueDisplayCoverLangEng}{0}
\newcommand{\ValueDisplayCoverLangChi}{1}
\newcommand{\VarDisplayCoverLang}{\ValueDisplayCoverLangEng}
\newcommand{\GetDisplayCoverLang}{\VarDisplayCoverLang}
\newcommand{\DisplayCoverInChi}{\renewcommand{\VarDisplayCoverLang}{\ValueDisplayCoverLangChi}}
\newcommand{\DisplayCoverInEng}{\renewcommand{\VarDisplayCoverLang}{\ValueDisplayCoverLangEng}}

% 日期顯示中文數字
\newcommand{\ValueDisplayCoverDateNumInNum}{0}
\newcommand{\ValueDisplayCoverDateNumInChi}{1}
\newcommand{\VarDisplayCoverDateNum}{\ValueDisplayCoverDateNumInNum}
\newcommand{\GetDisplayCoverDateNum}{\VarDisplayCoverDateNum}
\newcommand{\CoverDateNumInChi}{\renewcommand{\VarDisplayCoverDateNum}{\ValueDisplayCoverDateNumInChi}}

% ----------------------------------------------------------------------------

% Display Chinese and English name in english cover
\newcommand{\CoverDisplayNameChiEng}{0} % Default not display both
\newcommand{\SetCDBothName}{\renewcommand{\CoverDisplayNameChiEng}{1}}
\newcommand{\GetCDBothName}{\CoverDisplayNameChiEng}
\newcommand{\DisplayCoverPeoplesBothNames}{\SetCDBothName}

% A wrapper to handle \CDBothName{}
\newcommand{\CDBothName}{\DisplayCoverPeoplesBothNames}

% ----------------------------------------------------------------------------

% 顯示 '(初稿)' (中文版) 和 '(Draft)' (英文版) 在封面
\newcommand{\GetTextDraftChi}{(初稿)}
\newcommand{\GetTextDraftEng}{(Draft)}
\newcommand{\VarCoverDisplayDraft}{0} % Don't display in default
\newcommand{\EnableFlagDisplayDraft}{\renewcommand{\VarCoverDisplayDraft}{1}}
\newcommand{\DisplayDraft}{\EnableFlagDisplayDraft}
\newcommand{\GetFlagDisplayDraft}{\VarCoverDisplayDraft}

% ----------------------------------------------------------------------------
