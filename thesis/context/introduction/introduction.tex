% ------------------------------------------------
\StartChapter{Introduction}{chapter:introduction}
% ------------------------------------------------

淺談VR、MR、AR 三種實境差異,如圖所示 \ref{fig:VR_MR_AR} 。VR:利用電腦技術模擬出一個立體、高擬真的3D空間,當使用者穿戴特殊顯示裝置 (VR 眼鏡) 進入後,會 產生好像處在現實中一般的錯覺 。在這空間中,操作者可以藉由控制器或鍵盤在這個虛擬的環境下穿梭或互動。MR:把現實世界與虛擬世界合併在一起,從而建立出一個新的環境以及符合一般視覺上所認知的虛擬影像,在這之中現實世界中的物件能夠與數位世界中的物件共同存在並且即時的產生互動。這就是Mixed Reality Game所帶來的魅力,你能感受到虛擬的回饋,也能與虛擬世界做互動。AR:一種將虛擬資訊擴增到現實空間中的技術,它不是要取代現實空間,而是在現實空間中添加一個虛擬物件,藉由攝影機的辨識技術與電腦程式的結合,當設定好的圖片出現在鏡頭裡面,就會出現對應的虛擬物件。

在2015年,微軟發表了HoloLens(Mixture Reality Glass)

\InsertFigure
[label = {fig:VR_MR_AR},
scale = 0.55,
caption = {VR, MR, AR and XR},
align = center]
{./context/introduction/images/VR_MR_AR.png}


% ------------------------------------------------
\EndChapter
% ------------------------------------------------
